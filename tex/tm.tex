\section{Task Management}
\subsection{Data Structure}

As a data structure for a task management an array of circular linked
lists is used. Each index within the array represents a priority. Only
12 priority levels are allowed for user's tasks:  from 0 to 11. Thus,
the used array of circular linked lists contains 12 items. The item
within the array is the circular linked list of tasks. A priority of the
tasks equals to the index of the circular linked list stored these
tasks.

\begin{figure}[h]\centering\[\xymatrix{
	P\!A_{11} \ar@/_/[d] 
	& P\!A_{10} \ar@/_/[d] 
	& \cdots 
	& P\!A_{0} \ar@/_/[d] 
	\\
	\tau_{l} \ar@/^/[d] 
	& \tau_{l\!f} \ar@(dl,dr) 
	& 
	& \tau_{l} \ar@/^/[d]
	\\
	\tau_{f} \ar@/^/@{-->}[u] 
	& 
	& 
	& \tau_{f} \ar@/^/@{-->}[u]}\]
  \caption{Data structure for a task management}\label{fig:tmds}
\end{figure}

\begin{figure}[h]
	\centering
	\begin{subfigure}[h]{0.3\textwidth}
		\[
		  	\xymatrix{
				\tau_1  \ar@/^/[r]
				&\tau_2 \ar@/^/[ld] \\
				\tau_3 \ar@/^/[u]}
		\]
		\caption{1st graph}\label{fig:1st_graph}
	\end{subfigure}
	~
	\begin{subfigure}[h]{0.3\textwidth}
		\[
			\xymatrix{
				\tau_1  \ar@/^/[r]
				&\tau_2 \ar@/^/[ld] \\
				\tau_3 \ar@/^/[u]}
		\]
		\caption{2nd graph}\label{fig:2nd_graph}
	\end{subfigure}
	\caption{Simple graphs}\label{fig:graphs}
\end{figure}


We define the data structure for the task management as a priority array
of pointers to the circular queue of the tasks (we will denote it as
"rQi", where "i" is a priority of the given queue). Each queue
corresponded to the priority. Maximum number of tasks is 12, then
possible priorities are from 1 to 12 (and 0 for IDLE task).

Tasks with the same priorities are executed in a FIFO manner. During
execution we move E and pAi to the next task in the rQi. rQi always
points to the task, which is behind the executing task. If rQi is empty,
then pAi = NULL.

\begin{itemize}
    \item pA - priority array;
    \item pAi - pointer to the rQi (i-th element of the pA);
    \item E - pointer to the running task.
    \item Ti - i-th task.
\end{itemize}

\subsection{Insertion}

In general, task can be inserted into the rQi in a two different ways.
Insertion from rQi to rQi:

Insertion can be done for O(1), because we need to insert the task just
after *pAi due to FIFO execution.

\subsection{Deletion}

In general, task can be deleted in a several ways. Task deletes itself
(can be done for O(1)):


This deletion can be done for O(n) in the {\textbf{\textit{worst
case}}}. Last case, rQi$\rightarrow$rQl last task deletion, can be done
as in the "last task itself deletion" case, but without {\tt
context\_restore()} in the end. Note that tasks can not be deleted when
they hold a semaphore. To check this, the task needs to keep track of
whether it has locked any semaphores. To this end, a counter field was
added to the taskHandle structure, which counts the number of semaphores
the task has locked.


